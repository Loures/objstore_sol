% !TEX program = pdflatex
% !TEX options = -shell-escape -synctex=1 -interaction=nonstopmode -file-line-error "%DOC%"

\documentclass[a4paper,11pt]{article}
\usepackage[T1]{fontenc}
\usepackage[utf8]{inputenc}
\usepackage[italian]{babel}
\usepackage[bookmarks]{hyperref}
\usepackage{minted}
\usepackage{amsmath}
\usepackage[titletoc,title]{appendix}
\usepackage[ampersand]{easylist}

\usemintedstyle{emacs}
\hyphenation{lista double-linked}
\hyphenation{linkata}
\hyphenation{thread}
\hyphenation{ges-tione}

\begin{document}
\begin{titlepage}

\title{{Relazione progetto Laboratorio di\\Sistemi Operativi\\a.a. 2018-19}}
\author{Lorenzo Rasoini}
\date{}
\maketitle

\thispagestyle{empty}
\tableofcontents
\end{titlepage}

\newpage

\section{Scelte progettuali}
%\addcontentsline{toc}{section}{Scelte progettuali}
Di seguito vengono presentate le varie scelte progettuali effettuate durante la realizzazione del progetto.
\subsection{Strutture dati d'appoggio}
L'unica struttura dati utilizzata all'intero del progetto è \texttt{linkedlist} (lista double-linked concorrente) implementata all'interno
di \texttt{linkedlist.c}

Come per tutte le componenti del progetto la implementazione viene linkata all'eseguibile principale come file oggetto.
\subsubsection{\texttt{linkedlist}}
La lista double-linked è stata scelta per memorizzare i client connessi al server in un dato istante.
La scelta è stata dovuta alla facilità di implementazione e all'interesse che mi ha suscitato una sua implementazione in un contesto di programmazione concorrente.
Le procedure di ricerca \texttt{linkedlist\_search} e cancellazione \texttt{linkedlist\_delete} sono state dotate di un puntatore
a una procedura esterna che definisce, rispettivamente, i criteri per la ricerca e i criteri per la cancellazione.


\subsubsection{Gestione della concorrenza in \texttt{linkedlist}}
Per gestire accessi e scritture all'interno della lista è stato implementato un meccanismo di \emph{fine-grained locking}, ritenuto più efficente relativamente
all'utilizzo che viene fatto della lista all'interno del server, questa ipotesi è poi stata confermata dal confronto dei tempi di esecuzione con una implementazione dotata di un meccanismo di \emph{coarse-grained locking}.

Specificatamente, questo meccanismo è implementato tramite l'utilizzo di una lock: \texttt{mtx}, che blocca un singolo elemento della lista, e due puntatori a lock:
\texttt{prevmtx}, che punta alla lock dell'elemento precedente della lista (se esiste), e \texttt{nextmtx}, che punta alla lock dell'elemento successivo alla lista (se esiste);
le lock vengono sempre acquisite e rilasciate nello stesso ordine in modo tale da evitare situazioni di deadlock.

L'unica accortezza da avere nell'utilizzo di questa implementazione è l'allocare i dati puntati da \texttt{ptr} nell'heap del processo.
\newpage
\subsection{Strutturazione del codice}
Il codice del server è strutturato in moduli che corrispondono a un gruppo di funzionalità separate richieste dal server:
\begin{itemize}
    \item \texttt{os\_server.c}
    
    Main del server e i meccanismi di gestione delle interruzioni.
    \item \texttt{dispatcher.c}
    
    Codice relativo al thread dispatcher, e meccanismi di gestione per il segnale \texttt{SIGUSR1}.
    \item \texttt{worker.c}
    
    Codice relativo ai worker threads e ai meccanismi di ricezione e parsing dei messaggi inviati dai client.

    \item \texttt{os\_client.c}
    
    Definizione di procedure per la gestione dei comandi individuati dal parser.
    
    \item \texttt{fs.c}
    
    Definizione di procedure inerenti la scrittura e lettura dal file system dei dati inviati dai client e dei dati da inviare ai client.
    
    \item \texttt{linkedlist.c}
    
    Implementazione di una lista double-linked concorrente.
\end{itemize}
\smallskip
Il codice dell'interfaccia è interamente contenuto all'interno di \texttt{objstore.c}.

L'header file \texttt{errormacros.h} contiene varie macro utilizzate per la gestione di eventuali errori lanciati da parte di chiamate di sistema.
\section{Funzionamento generale}
%\addcontentsline{toc}{section}{Funzionamento generale}
%\renewcommand*{\theHsection}{chY.\the\value{section}}
%\setcounter{section}{1}
\subsection{Interazione intra-processo}
Il thread principale, dopo aver creato la socket e aver messo in piedi i meccanismi di gestione dei segnali, crea un thread dispatcher, il quale
si occupa di gestire le connessioni che arrivano alla socket. Ogni qualvolta si presenti una richiesta di connessione su quest'ultima il thread dispatcher
valuta se il server è in condizione di accettarla o meno e, in caso positivo, accetta la connessione e crea il thread worker destinato a gestire tutte le interazioni server-client fino
alla disconnessione di quest'ultimo.
\subsubsection{Comunicazione intra-processo}
La comunicazione intra-processo è basata su l'utilizzo di due variabili condivise: \texttt{OS\_RUNNING}, che segnala a tutti i thread del processo la terminazione
dello stesso, e \texttt{worker\_num}, dove al suo interno è memorizzato il numero di client connessi.
\subsubsection{Gestione della terminazione}
Il processo di gestione della terminazione ha inizio immediatamente dopo l'arrivo di un segnale \texttt{SIGTERM} o \texttt{SIGINT}; in seguito alla ricezione di tale segnale
il thread adibito alla gestione dei segnali (il thread principale) setta \texttt{OS\_RUNNING} a 0.

A questo punto ogni thread worker rileverà il cambiamento della variabile, uscirà dal suo loop e darà inizio alla procedura di cleanup,
al termine della quale il thread segnalerà al thread dispatcher l'avvenuta terminazione tramite la variabile di condizione \texttt{worker\_num\_cond}.

Appena \texttt{worker\_num} diventa 0 il thread dispatcher esegue la propria procedura di cleanup, termina e restituisce il controllo al thread principale che farà terminare il processo.
\subsection{Gestione dei segnali}
La gestione dei segnali \texttt{SIGINT} e \texttt{SIGTERM} avviene interamente all'interno del thread principale, il quale, dopo aver creato il dispatcher, si mette in attesa dei due segnali.

Per quanto riguarda la gestione di \texttt{SIGUSR1} è stata usata la tecnica del \emph{self-pipe trick}: il thread principale scrive su una pipe letta dal thread dispatcher; quest'ultimo
una volta rilevata la presenza di dati nella pipe la svuota ed esegue la procedura \texttt{stats}.

\subsection{Gestione della memoria}
Per quanto concerne la gestione della memoria si è cercato di usare il più possibile lo stack di ciascun thread e usare le allocazioni sullo heap solo per dati
di natura dinamica o di grandi dimensioni (in modo da scongiurare situazioni di stack overflow).
\newpage
\begin{appendices}
\section{Sistemi operativi usati per il testing}
\begin{itemize}
    \item Arch Linux con kernel Linux versione 5.1.15
    \item Xubuntu 14.10 (macchina virtuale)
    \item Ubuntu 16.04 LTS
    \item macOS 10.14 Mojave
\end{itemize}

\section{Test aggiuntivi}
\texttt{make bigblock} e \texttt{make interactive} (quest'ultino richede la presenza di \emph{GNU Readline} nel sistema)
producono due ulteriori client di test per il progetto.

Il comando \texttt{./bigblock <nome>} si registra come \texttt{nome} al server e testa la lettura/scrittura di un blocco di dati di 100MB.
Per le istruzioni riguardanti \texttt{interactive} leggere il file di testo \texttt{interactive.txt}. 
(N.B. questi due test richiedono che una istanza del server sia già in esecuzione)
\section{Elemento della lista}
\begin{listing}[ht]
    \begin{minted}{C}
    typedef struct linkedlist_elem {
        void *ptr;
        struct linkedlist_elem *prev;
        pthread_mutex_t *prevmtx;
        pthread_mutex_t mtx;
        pthread_mutex_t *nextmtx;
        struct linkedlist_elem *next;
        } linkedlist_elem;
    \end{minted}
    \caption{\texttt{struct linkedlist\_elem}}
\end{listing}
\end{appendices}    

\end{document}